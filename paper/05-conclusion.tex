\section{Conclusion}
We have presented the idea and results for the incorporation of atrous convolutions into Godard \etal's~\cite{Godard_2017_CVPR} unsupervised monocular depth estimation framework using left-right consistency.
Experiments with atrous convolutions were mainly conducted using the Atrous Spatial Pyramid Pooling block, inserted between the encoder and decoder, which was already successful in the task of image segmentation~\cite{chen2018deeplab}.
We conclude, that atrous convolutions within the presented experiments do not improve monocular depth estimation. Furthermore, the use of atrous convolutions enforces a higher output stride which consequently harms runtime and memory consumption.

In the event of these experiments we additionally found out, that in Godard \etal's~\cite{Godard_2017_CVPR} architecture, it is possible to reduce the number of channels between the encoder and decoder. This can decrease the number of network parameters and improve runtime, without losing predictive power.

Our experiments focused on architectural design instead of hyperparameter tuning. Finding the optimal set of hyperparameters for models with atrous convolutions might lead to different results. Future work should investigate why atrous convolutions work so well in semantic segmentation, but cannot be applied straightforwardly to continuous depth estimation. In semantic segmentation, there is a clear distinction between object classes, while regression problems face the harder problem of predicting a continuous signal. Recent work suggests that atrous convolutions help if depth estimation is treated as a classification problem with discrete depths \cite{Fu2018}. Alternatively, semantic features could be leveraged to improve depth estimation performance \cite{Jiao2018}, in which case atrous convolutions might be helpful.
